%%%%%%%%%%%%%%%%%%%%%%%%%%%%%%%%%%%%%%%%%
% Medium Length Professional CV
% LaTeX Template
% Version 2.0 (8/5/13)
%
% This template has been downloaded from:
% http://www.LaTeXTemplates.com
%
% Original author:
% Trey Hunner (http://www.treyhunner.com/)
%
% Important note:
% This template requires the resume.cls file to be in the same directory as the
% .tex file. The resume.cls file provides the resume style used for structuring the
% document.
%
%%%%%%%%%%%%%%%%%%%%%%%%%%%%%%%%%%%%%%%%%

%----------------------------------------------------------------------------------------
%	PACKAGES AND OTHER DOCUMENT CONFIGURATIONS 
%----------------------------------------------------------------------------------------
 
\documentclass{resume} % Use the custom resume.cls style 
\usepackage[dvipsnames, svgnames, x11names]{xcolor}  % 加载扩展颜色库
\usepackage[left=0.4 in,top=0.3 in,right=0.4 in,bottom=0.3in]{geometry} % Document margins
\newcommand{\tab}[1]{\hspace{.2667\textwidth}\rlap{#1}}
\newcommand{\itab}[1]{\hspace{0em}\rlap{#1}}

\name{\textcolor{MidnightBlue}{Rui GONG, PhD}} % Your name 
\address{D2, 1068 Xueyuan Avenue, Shenzhen University Town, Xili, Nanshan District, Shenzhen, China} % Your address 
\address{+86 18629245948 \\ rgongab@connect.ust.hk} % Your phone number and email

\definecolor{TsinghuaPurple}{cmyk}{0.58,0.90,0,0}
\renewenvironment{rSection}[1]{
\sectionskip
\textcolor{MidnightBlue}{\MakeUppercase{#1}}
\sectionlineskip
\hrule
\begin{list}{}{
\setlength{\leftmargin}{0em}
}
\item[]
}{
\end{list}
}

\begin{document}  

%----------------------------------------------------------------------------------------
%	EDUCATION SECTION
%----------------------------------------------------------------------------------------

\begin{rSection}{\textbf{Education}}  % 改变标题的颜色

{\bf Doctor of Philosophy, Physics} \hfill {September 2018 - March 2023}
\\ 
Hong Kong University of Science and Technology
\\ 
Co-supervisors:\textit{ Prof. Jiannong WANG}, Wu Chien-Shiung Professor of Science, Head and Chair Professor \\
\makebox[2.6cm][l]{\hspace{0pt}}\textit{ Prof. Hong WANG}, IEEE Fellow, Distinguished Young Scholars

{\bf Bachelor of Electronic Science and Technology} \hfill {September 2014 - June 2018}
\\ 
Xi'an Jiaotong University, Shaanxi 

%Minor in Linguistics \smallskip \\
%Member of Eta Kappa Nu \\
%Member of Upsilon Pi Epsilon \\


\end{rSection} 

%----------------------------------------------------------------------------------------
%	Current Position
%----------------------------------------------------------------------------------------

\begin{rSection}{\textbf{Current Position}}
{\bf Postdoctoral Researcher} \hfill {May 2023 - May 2025}
\\Institute of Technology for Carbon Neutrality, Shenzhen Institutes of Advanced Technology
\\Chinese Academy of Science
\\Postdoctoral Supervisors: \textit{Prof. Baofu DING} \& \textit{{Prof. Huiming CHENG}}

\end{rSection}
%----------------------------------------------------------------------------------------
%Research Interests
%----------------------------------------------------------------------------------------

\begin{rSection}{\textbf{Research Interests}}
My research focuses on \textbf{smart stimulus-responsive materials}, with a particular emphasis on perovskite-based and low-dimensional materials for optoelectronic and photonic applications. \\\textit{Selected:}
\begin{itemize}
    \item \textbf{Green Synthesis of Perovskite-based materials:} Development of multimodal-responsive perovskite/polymer composites for optical encryption, rewritable displays, and secure information storage.
\end{itemize}

 \begin{itemize}
     \item \textbf{Precise alignment of 1D/2D nanomaterials:} Innovation in 2D liquid crystal devices for ultra-sensitive electrochromic applications, enabling energy-efficient smart windows with vivid interference color.
 \end{itemize}

\end{rSection}

  
\begin{rSection}{\textbf{skills and INTERESTS}}

\begin{tabular}{ @{} >{\bfseries}l @{\hspace{6ex}} l }  
Lab Techniques:& XRD, SEM, AFM, TEM, XPS, PL, PLQY, UV-Vis, FT-IR, Raman; Electrospinning, \\
& Casting, Spin-coating, 3D printing\\
Softwares:& OriginPro, Photoshop, Adobe illustrator, CorelDraw, 3D Max, Camera4D, COMSOL\\    
Programming:& C/C++, Python, MATLAB\\
Language:& Mandarin (Native), English (Proficient), German (Basic)\\       
\end{tabular}   
\end{rSection}


%-------------------------------------------------

%--------------------------------------------------------------------------------------
%   Research Publications 
%--------------------------------------------------------------------------------------
\begin{rSection}{\textbf{Research Publications}} \itemsep -3pt        

{
\textbf{ Journal Articles}
\begin{itemize}
    \item \underline{\textbf{R. Gong}}, S. Tian, Y. Lei, Z. Zhang, Y. Xu, R. Lyu, F. Wang, H. Zhang, Z. Huang, C. Zhu, B. Liu*, B. Ding, Tunable pure interference colors of 2D titania liquid crystal with ultrasensitive electroresponse. \textit{\textit{Science Advances}} (Accepted).

\end{itemize}
\begin{itemize}
    \item \underline{\textbf{R. Gong}}, F. Wang, J. Cheng, Y. Lu, R. Hu, H. Huang, B. Ding, H. Wang, Hydrochromic Effect of Perovskite-Polymer Composites. \textit{\textit{ACS Nano}} \textbf{\textbf{18}}, 33097–33104 (2024).
    \item \underline{\textbf{R. Gong}}, F. Wang, J. Cheng, Z. Wang, Y. Lu, J. Wang, H. Wang, Weak-solvent-modulated optical encryption based on perovskite nanocrystals/polymer composites. \textit{\textit{Chemical Engineering Journal}} \textbf{\textbf{446}}, 137212 (2022).
    \item F. Wang, R. Lyu, H. Xu, \underline{\textbf{R. Gong}}, B. Ding, Tunable colors from responsive 2D materials. \textit{\textit{Responsive Materials}}, e20240007 (2024).
    \item T. Chen, X. Mai, Y. Li, T. Wang, \underline{\textbf{R. Gong}}, F. Chen, H. Huang, Z. Yan, F. Wang, Sustainable rewritable paper based on photoresponsive tungsten oxide quantum dots for anti-counterfeiting and waterproofing. \textit{\textit{Chemical Engineering Journal}} \textbf{\textbf{499}}, 155999 (2024).
    \item     H. Xu, J. Liu, S. Wei, J. Luo, \underline{\textbf{R. Gong}}, S. Tian, Y. Yang, Y. Lei, X. Chen, J. Wang, G. Zhong, Y. Tang, F. Wang, H.-M. Cheng, B. Ding, A multifunctional optoelectronic device based on 2D material with wide bandgap. \textit{Light: Science} \& \textit{Applications} \textbf{\textbf{12}}, 278 (2023).
    \item J. Dong, R. Hu, Y. Niu, L. Sun, L. Li, S. Li, D. Pan, X. Xu, \underline{\textbf{R. Gong}}, J. Cheng, Z. Pan, Q. Wang, H. Wang, Enhancing high-temperature capacitor performance of polymer nanocomposites by adjusting the energy level structure in the micro-/meso-scopic interface region. \textit{\textit{Nano Energy}} \textbf{\textbf{99}}, 107314 (2022).
    \item Z. Wang, J. Cheng, R. Hu, X. Yuan, Z. Yu, X. Xu, F. Wang, J. Dong,\underline{\textbf{R. Gong}}, S. Dong, H. Wang, An approach combining additive manufacturing and dielectrophoresis for 3D-structured flexible lead-free piezoelectric composites for electromechanical energy conversion. \textit{\textit{J. Mater. Chem. A}} \textbf{\textbf{9}}, 26767–26776 (2021).
\end{itemize}
\textbf{ Patents}
\begin{itemize}
    \item H. Wang, \underline{\textbf{R. Gong}}, F. Wang, A fluorescent anti-counterfeiting composite material and its preparation method, CN114561206B, Active.
    \item F. Wang, \underline{\textbf{R. Gong}}, B. Ding, A Film Material for Unclonable Anti-Counterfeiting Labels: Preparation Method and Applications, CN117700906A, Pending.
\end{itemize}



 }  
 
\end{rSection}


%	INTERNSHIP/TRAININGS 
%----------------------------------------------------------------------------------------

\begin{rSection}{\textbf{Projects}} \itemsep -3pt  

{\textbf{Guangdong Province Overseas Postdoctoral Talent Support}} \hfill 2023-2025 \\   
{\textbf{Key Laboratory of Intelligent Design and Application of Low-Dimensional Materials}, \\Participant} \hfill 2025\\     
\end{rSection}  
%---------------------------------------------------------------------------------
%  Achievements
%--------------------------------------------------------------------------------
\begin{rSection}{\textbf{Selected Honors and Awards}} 
\itemsep -2pt
Full Postgraduate Studentship, HKUST  \hfill{2018 - 2023}\\
Peng Kang Scholarship \& Excellent Student, XJTU \hfill{2015 - 2017}\\
Chinese Physics Olympiad (CPhO), Bronze medal \hfill{2013} 

\end{rSection}


\end{document}
